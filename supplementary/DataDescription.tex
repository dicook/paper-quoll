\documentclass{article}

\usepackage{graphicx}
\usepackage{hyperref}
\usepackage{multirow}

\begin{document}

\title{Data description}
%\author{Author's Name}

\maketitle


The following document contain a short description of the datasets used for the data competitions described in the paper.

\section{Melbourne properties price dataset}

The dataset `\textit{Melbourne properties price}' contains sales prices of different types of residential properties in Melbourne area along with other characteristics of the properties. 

The variables and their descriptions prevented in table \ref{tab:Melb_price}:

\begin{table}[h]
	\begin{center}
		\renewcommand{\arraystretch}{1.1}
		\begin{tabular}{p{2cm} p{13cm}} \hline
			Variable   				& Description       \\\hline
			price 		& Price house sold for in AUS dollars\\
			suburb 		& Suburb: different areas of Melbourne\\
			land\_size  & Area of the land in square meters \\
			house\_size & Area of the house or the apartment in square meters\\
			nbeds 		& Number of bedrooms\\
			nbaths 		& Number of bathrooms/toilets\\
			ncars 		& How many cars fit in the carport/garage\\
			result 		& S - property sold; SP - property sold prior; PI - property passed in; VB - vendor bid; SA - sold after auction\\
			agent 		& Selling agent\\
			property\_type & h house; u appartment; t town-house or unit \\
			nvisits 	& Number of visitors before the selling\\
			rating 		& The (rounded) average rating by the visitors, 0 to 10, 0-``awful'' to 10-``unbelivebly lovely'' \\
			year 		& YYYY Year sold \\
			month 		& MM Month sold\\
			day 		& DD day sold\\
			id 			& Unique id for each house \\\hline
		
	\end{tabular}
	\caption{Melbourne properties price dataset: variable description.}
\end{center}
\label{tab:Melb_price}
\end{table}


The data were collected between Feb 2, 2013 and Dec 17, 2016. The training set ans the test set contained 75,366 and 25,122 observations, respectively.  
The associated errors were calculated as absolute errors: $\frac{1}{N} \sum_{i=1}^{N} |y_i - \hat{y}_i|$. 






\section{Spam dataset}
The `\textit{Spam}' data contains few thousands emails that were classified by their
owners as either spam or non-spam (3066 in the training set and additional 1533 in the testing set). Each email message also includes many other characteristics.

The variables and their descriptions prevented in table \ref{tab:spam_var}:

\begin{table}[h]
	\begin{center}
		\renewcommand{\arraystretch}{1.2}
		\begin{tabular}{p{4cm} l} \hline
		Variable   				& Description       \\\hline
		numRecipients		    & The number of people in the To: or Cc: lines\\
		Domain      		    & Domain name of the sender's email address\\
		replyToSameAddress      & TRUE/FALSE\\
		Weekday 				& Day of the week\\
		Hour 					& Hour of the day\\
		percentLowerCase 		& \% of lower case letters in the subject line\\
		numDigits 				& Number of digits in the sender's email address\\
		percentNonLetters 		& \% of non-letters in the sender's email address\\
		Size 					& Size of the email in kb\\
		numLinks				& Integer number of links included in the email\\
		localSender 			& Indicator - was the sender in the local domain as the\\
		receiver
		credit, porn, sucker,
		pharm, prescription,
		drugs, save, sex, dis-
		creet, free, sell, sale,
		asseenon, discount		& Binary variables computed by the presence of
		certain key words \\
		newsletter 				& Indicator - is this email basically a newsletter\\
		Spam 					& Indicator - classification of the email as spam or not by the receiver\\
		id 						& Unique id for each email \\\hline
			
		\end{tabular}
		\caption{Spam dataset: variable description.}
	\end{center}
	\label{tab:spam_var}
\end{table} 

The mean consequential error (MCE) was selected as the method for prediction accuracy measurement. The mean/average of the "consequential error", where all errors are equally bad is given by:  
$$ MCE = \frac{1}{N} \sum_{y_i \neq \hat{y}_i} 1$$

%\begin{figure}
%    \centering
%    \includegraphics[width=3.0in]{myfigure}
%    \caption{Simulation Results}
%    \label{simulationfigure}
%\end{figure}







\end{document}